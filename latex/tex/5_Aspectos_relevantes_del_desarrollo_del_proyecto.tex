\capitulo{5}{Aspectos relevantes del desarrollo del proyecto}
En este apartado se expondrán los aspectos relevantes del desarrollo del proyecto, definiendo los problemas y las decisiones tomadas.

\section{Bibliotecas de clustering}
Para el desarrollo del proyecto se necesitaban bibliotecas que implementaran algoritmos de \emph{clustering}. La mayoría de bibliotecas en Java tenían una licencia de uso incompatible a MIT (licencia de la aplicación). También se ha investigado sobre bibliotecas de otros lenguajes como Scala y Kotlin, pero se han descartado por el desconocimiento del lenguaje. Por lo tanto, en el proyecto se ha enfocado a dos bibliotecas: la de \href{http://commons.apache.org/proper/commons-math/userguide/ml.html}{Apache Commons Math} en su apartado de \emph{Machine Learning} y la de \href{https://haifengl.github.io/clustering.html}{Smile} en su apartado de \emph{clustering}.

\section{Selección de parámetros}
Cada algoritmo de \emph{clustering} tiene diferentes parámetros de ejecución, por lo que es necesario indicar al usuario qué parámetros. En JavaFX no se incluye ningún componente que permita introducir datos en formato clave-valor. 

Se han planteado dos soluciones: la primera es añadir componentes de selección (campo de texto, cuadro de elección) de forma dinámica y en función del algoritmo seleccionado mostrar los parámetros correspondientes. Esta solución requiere programar de forma primitiva la creación y gestión de los componentes.
La otra alternativa es utilizar el componente \emph{PropertySheet} de la librería de extensión, ControlsFX, el cual muestra los parámetros en formato clave-valor. También permite la modificación de forma dinámica de los elementos. Esto nos permite mostrar los parámetros necesarios para cada algoritmo.

\imagen{memoria/Kmeans}{\emph{PropertySheet} para el algoritmo \emph{KMeans}.}
\imagen{memoria/FuzzyKmeans}{\emph{PropertySheet} para el algoritmo \emph{FuzzyKMeans}.}

Se ha optado por resolver el problema con esta última alternativa.

\section{Visualización del resultado del clustering}
Para mostrar al usuario, el resultado del \emph{clustering} se ha implementado la visualización en forma de tabla y en una gráfica de dispersión, tanto en 2D como en 3D.

En la gráfica de dispersión, a cada punto le corresponde unas coordenadas de tamaño 2 o 3 según la dimensionalidad de la gráfica. Como los datos utilizados para realizar el \emph{clustering} son de $n$ dimensiones, siendo $n$ el número de elementos seleccionados por el usuario, es necesario reducir la dimensión de los datos para que puedan ser mostrados en la gráfica.

Para resolver este problema se han empleado técnicas de reducción de dimensiones, ademas se ha investigado sobre las bibliotecas en Java que las implementan. Se han valorado el método de PCA y el de t-SNE y los resultados obtenidos son bastante similares, pero la velocidad del método t-SNE es mucho menor, por lo que se ha optado por la implementación de PCA.

\section{Análisis de silueta}
La idea principal era buscar una biblioteca en Java que implementara el algoritmo del análisis de silueta. Como no se encontraron opciones válidas, se plantearon dos alternativas, buscar una biblioteca en otro lenguaje y tratar de ejecutarlo desde Java o hacer nuestra propia implementación del algoritmo.

Se optó por la segunda opción, ya que el algoritmo no es muy complejo y si utilizábamos una implementación externa habría que adaptar la estructura de los datos. Para la implementación del algoritmo se siguió el pseudocódigo de \href{https://www.datanovia.com/en/lessons/cluster-validation-statistics-must-know-methods/#silhouette-coefficient}{DataNova}.

\section{Diagrama de dispersión 3D}
Para la implementación del diagrama de dispersión 3D se han considerado diferentes opciones. Una de ellas era buscar algún componente de JavaFX que permitiera realizarlo de forma sencilla, otra opción era buscar una biblioteca en JavaScript e incorporarlo a un \emph{WebView} de JavaFX.

La primera opción se descartó porque requería una programación de muy bajo nivel, especificando la posición exacta de cada elemento de la gráfica (ejes, cuadrículas, leyendas, puntos o esferas).

Para la segunda opción se contemplaron varias bibliotecas, una de ellas es \href{https://plotly.com/}{plotly} que tenía una implementación específica para clústeres y muy atractiva, se puede observar un ejemplo en la figura \ref{fig:memoria/plotly}. Se descartó esta opción porque el componente \emph{WebView} de JavaFX no soportaba el motor gráfico WebGL. Otras bibliotecas similares también tenían este problema.

Finalmente se optó por utilizar la biblioteca Highcharts que incluye el diagrama de dispersión, pero algo menos interactivo que las anteriormente valoradas. Se puede observar un ejemplo en la figura \ref{fig:memoria/highcharts}.

\imagenConTamano{memoria/plotly}{Gráfico de clústeres 3D de plotly.}{0.7}

\imagenConTamano{memoria/highcharts}{Ejemplo del diagrama de dispersión 3D de highcharts.}{0.7}

\section{Aleatoriedad de los resultados}
Ejecutar algunos algoritmos con los mismos parámetros y datos generan resultados distintos. Esto se debe a que algunos algoritmos utilizan un RNG (\emph{random number generator}) en algún paso de la ejecución. Esto producía que el resultado del \emph{clustering} fuese mejor o peor aleatoriamente.

Para solucionar este problema se habilitó la posibilidad de que el usuario eligiera el número de iteraciones del algoritmo. Y para determinar lo bueno que es el \emph{clustering} se empleó el coeficiente de silueta.

Al aplicar esta solución surgió un nuevo problema, la ejecución de muchas iteraciones bloqueaba el hilo principal de JavaFX y congelaba la aplicación. Para resolver esto se envió la ejecución a otro hilo utilizando la clase \emph{Service} de JavaFX.

\section{Nombre de los clústeres}
Al ejecutar los algoritmos de \emph{clustering} el resultado es obtenido en una lista de clústeres, donde el identificador del clúster es el índice que ocupa en la lista. En las visualizaciones del \emph{clustering} se utiliza este número para reconocer a cada clúster.

Para resolver este problema se ha dado la posibilidad al usuario de renombrar o etiquetar el nombre de cada clúster.
\imagen{memoria/etiquetar}{Solución al nombre de los clústeres}

\section{Transformación y filtrado de datos}
Después de que el usuario haya elegido los datos con los que realizar el \emph{clustering} hay que transformar los datos, ya que no tiene mucho sentido comparar calificaciones en un escala de 0 a 100 con el total de registros de un alumno.

Para solucionar esta cuestión, se ha realizado una normalización lineal de los datos a una escala de 0 a 1.

$X' = \frac{X - X_{min}}{X_{max} - X_{min}}$

En cuanto al filtrado de datos, puede ser razonable eliminar aquellos valores que son constantes a todos los alumnos, aunque con algunas pruebas realizadas el \emph{clustering} es a veces mejor con estos valores. Se ha optado por dejar esta opción a elección del usuario.

\section{Clustering jerárquico}
Para la realización del \emph{clustering} jerárquico se planteo inicialmente utilizar la biblioteca \href{https://github.com/lbehnke/hierarchical-clustering-java}{hierarchical-clustering-java}, la cual daba una implementación bastante sencilla de utilizar. Una vez integrada esta biblioteca en la aplicación se procedió a valorar las posibles representaciones del resultado obtenido en el \emph{clustering} jerárquico.

Para la visualización se contemplaron dos opciones: la primera era utilizar el propio dendrograma que ofrecía la biblioteca en Java Swing \ref{fig:memoria/dendrograma1}, la segunda opción era coger los datos numéricos del resultado y utilizar una biblioteca de gráficas para su representación.
\imagen{memoria/dendrograma1}{Dendrograma de la biblioteca.}

Para la segunda opción se realizó una implementacion empleando la biblioteca de Chart.js con la extensión \href{https://github.com/chartjs/awesome}{awesome} que ofrecía los dendrogramas.
\imagen{memoria/dendrograma2}{Dendrograma de implementación propia.}

Inicialmente se opto por esta última opción, aunque no es muy buena, ya que no se podían ajustar los nodos en función de la distancia entre clústeres. Posteriormente se incluyo en la aplicación otra biblioteca de \emph{clustering}, el cual incluía una implementación del \emph{clustering} jerárquico aglomerativo.

La biblioteca \textbf{smile} permite ejecutar el \emph{clustering} jerárquico definiendo la medida de distancia entre punto y entre agrupaciones. El resultado obtenido es mostrado en una imagen que contiene el dendrograma.
\imagen{memoria/dendrograma3}{Dendrograma de la biblioteca de smile.}

Se ha optado por continuar con esta opción ya que ofrece más posibilidades, aunque tiene un inconveniente, en el dendrograma no se muestra el nombre del alumno en la rama. Se ha solucionado este problema integrando la partición del \emph{clustering} jerárquico en un número determinado de agrupaciones.
\imagen{memoria/particion}{Opción de dividir el resultado en agrupaciones.}

\section{Calidad de código}
Para establecer que el código programado cumpla unos mínimos de calidad, se han empleado herramientas de análisis de código. Las herramientas utilizadas son \href{https://www.codacy.com/}{Codacy} y \href{https://codeclimate.com/}{Code Climate}. Los resultados obtenidos por Codacy.
\imagen{memoria/codacy}{Análisis por Codacy.}
Se observa que se tienen 0\% de \emph{Issues}, el cual indica que no se tiene defectos de código o \emph{code smells}. La complejidad de los fichero indica la complejidad ciclomática de cada clase. Y la duplicidad de código es el porcentaje de código repetido que hay.

Y los resultado de Code Climate.
\imagen{memoria/codeClimate}{Análisis por Code Climate.}
En el analisis se muestra en número de \emph{code smells} y los bloques de código duplicado que hay. La mantenibilidad se calcula en función del tiempo necesario para corregir los defectos.


Cabe destacar que solo se han analizado los ficheros Java referentes a la parte de \emph{clustering} y se han ignorado el resto de ficheros que no forman parte de este proyecto.
