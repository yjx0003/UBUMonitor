\apendice{Especificación de Requisitos}

\section{Introducción}

En este anexo se comentará los objetivos generales del proyecto y detallando sobre los requisitos funcionales y no funcionales establecidos durante el desarrollo del proyecto.

\section{Objetivos generales}
El proyecto tiene como objetivos la refactorización del código junto con la integración de los registros del curso. Añadir nuevas mejoras de interacción con el usuario así como poder guardar los datos en ficheros cifrados. También realizar nuevas gráficas de los registros junto con sus opciones de filtrado.

\section{Catalogo de requisitos}

Estos son los siguientes requisitos funcionales y no funcionales:

\subsection{Requisitos funcionales}

\begin{itemize}
	

	\item \textbf{RF-1 Guardar los datos:} la aplicación tiene que ser capaz de guardar los datos para sus posteriores ejecuciones.
	\begin{itemize}
		\item \textbf{RF-1.1 Almacenar en un fichero:} la aplicación debe almacenarlo en archivos locales.
		\item \textbf{RF-1.2 Cifrar los datos:} la aplicación debe encriptar los datos cuando se guarde.
	\end{itemize}
	\item \textbf{RF-2 Recordar datos sesión:} la aplicación debe dar la opción de recordar el usuario y \textit{host} en sucesivas ejecuciones.
	\item \textbf{RF-3 Opción de actualizar:} el usuario debe tener la posibilidad de elegir 
	\item \textbf{RF-3 Filtros de visualización:} en la visualización de los registros tenga posibilidad de filtrar.
	\begin{itemize}
		\item \textbf{RF-3.1 Filtrado por agrupaciones:} filtros agrupando por diferentes formas de tiempo.
		\item \textbf{RF-3.2 Filtrado por fechas:} filtros según fechas del calendario.
	\end{itemize}
	\item \textbf{RF-4 Aspectos de visualización de los usuarios:} mejorar la visualización de los participantes del curso.
		\begin{itemize}
		\item \textbf{RF-4.1 Imágenes de los usuarios:} mostrar los fotos de los usuarios asociados en Moodle.
		\item \textbf{RF-3.2 Últimos accesos:} mostrar los últimos accesos de los usuarios a la plataforma de Moodle.
	\end{itemize}
	\item \textbf{RF-5 Integrar con otros proyectos:} debe poder integrar lo máximo posible de los proyectos UBUGrades y UBULogs.s
	\item \textbf{RF-6 Fechas de los registros:} los registros se deben guardar en la zona horaria del servidor, hay que transformar a la del servidor si el usuario está usando otra zona horaria .
	
	
\end{itemize}

\subsection{Requisitos no funcionales}

\begin{itemize}
	\item \textbf{RNF-1 Tiempos de respuesta:} la aplicación debe tener buenos tiempos de respuesta a las diferentes acciones del usuarios. 
	\item \textbf{RNF-2 Gestión de memoria:} el consumo de memoria de la aplicación no debe ser excesiva.
	\item \textbf{RNF-3 Usabilidad:} el esfuerzo del usuario para aprender a usar la aplicación debería ser mínimo.
	\item \textbf{RNF-4 Soporte:} brindar al usuario una fácil instalación, documentaciones de uso y facilidad de actualización hacia versiones más recientes.
	\item \textbf{RNF-5 Internacionalización:} la aplicación debe poder añadir fácilmente otros idiomas.
\end{itemize}


\section{Especificación de requisitos}
La aplicación solo posee un actor en los casos de uso, el usuario.
\subsection{Diagrama de casos de uso}
\imagen{diagrama_casos_uso}{Diagrama de casos de uso.}