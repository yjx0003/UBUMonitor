\capitulo{3}{Conceptos teóricos}

En este apartado comentaremos sobre conceptos que aparecen en Moodle.



\section{Conceptos de Moodle}
\subsection{Moodle}



\subsection{Módulo del curso}
El módulo del curso\cite{noauthor_course_nodate} (en inglés \textit{course module}) también abreviado como `cm', representa las actividades y recursos del curso. Estos contienen información sobre qué curso y sección se muestra, además de la configuración de visibilidad y otros datos relevantes. Los identificadores únicos de los módulos del curso se definen como \textbf{`cmid'}. 
\imagen{course_modules}{Ejemplos de módulos del curso que actividades y recursos.}

\subsection{Ítem de calificación}
El Ítem de calificación\cite{noauthor_grade_nodate} (en inglés \textit{Grade item}), es cada fila de la tabla de calificaciones.
Las carpetas de la tabla, a parte de Ítem de calificación, se considera también como categoría (\textit{category}).
Los

\imagen{item_calificacion}{Ejemplo de una tabla con varios Ítem de calificación.}

\subsection{Categorías de cursos}
Las categorías de cursos\cite{noauthor_course_nodate-1} (en inglés \textit{course categories}) organizan los cursos de la página de Moodle de forma jerárquica.

\imagen{course_categories}{Ejemplo de categorías del curso enmarcados en rojo.}

\section{Otros términos}
\subsection{REST API}
REST\cite{noauthor_api_nodate} son todas las interfaces entre sistemas que use HTTP para conseguir datos o realizar operaciones sobre esos datos en todos los formatos posibles, como puede ser XML y JSON.

En Moodle la funcionalidad de REST API se conoce como \textit{Web Services} que a partir de un \textit{token} generado previamente por la función \textbf{\textit{moodle\_mobile\_app}} se puede acceder a otras funciones del servicio. Puede ser desde ver las calificaciones de alumnos hasta crear actividades o tareas.



